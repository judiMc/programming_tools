%% LyX 2.2.2 created this file.  For more info, see http://www.lyx.org/.
%% Do not edit unless you really know what you are doing.
\documentclass[12pt,english]{article}
\renewcommand{\familydefault}{\sfdefault}
\usepackage[latin9]{inputenc}
\usepackage{geometry}
\geometry{verbose,tmargin=1in,bmargin=1in,lmargin=1cm,rmargin=1in}
\usepackage{babel}
\usepackage{url}
\usepackage{amsmath}
\usepackage[unicode=true]
 {hyperref}

\makeatletter
%%%%%%%%%%%%%%%%%%%%%%%%%%%%%% User specified LaTeX commands.
\date{ }



\usepackage{babel}
\usepackage{listings}
\renewcommand{\lstlistingname}{Listing}

\makeatother

\usepackage{listings}
\renewcommand{\lstlistingname}{Listing}

\begin{document}

\title{Review: Testing}
\maketitle

\section*{Overview}

Software development requires constant debugging and testing code
to ensure quality and functionality of the software. Proper testing
includes multiple forms of testing, the most common is unit testing.
This review will introduce the benefits of unit testing, and the concepts
of code coverage and edge-case testing. This review includes a practical
example of developing an integer array container, and how to properly
test your code to ensure it's functionality.

\section*{Learning Outcomes \label{sec:Learning-Outcomes}}

Upon successful completion of this review, you should be able to: 
\begin{itemize}
\item comprehend the concepts of unit testing, code coverage, and edge testing
\item identify components of software code that require testing
\item apply proper testing patterns to ensure code quality.
\end{itemize}

\section*{Introduction \label{sec:Introduction}}

One of the most time-consuming aspects of software development is
manually testing or walking through test-cases of your code to ensure
correctness. As software projects and team sizes grow it increases
the potential for bugs to be accidentally introduced. By creating
small automated tests known as unit tests, you can ensure that fewer
bugs will be accidentally introduced, that your code is functioning
as expected, and can assure other developers you've created quality
software.

Unit testing focuses on testing the smallest functional unit of code,
this doesn't mean testing every line or every section of code but
focuses testing on specific sections of grouped functionality. Deciding
what requires unit testing is a significant part of ensuring quality,
usefulness, and an appropriate cost to benefit ratio. If you are developing
an abstract data type, such as a stack or list, generally you will
have a unit test for each operation such as create, insert, remove,
destroy and others. Unit testing should be separated from the main
program and executable via the makefile. The benefit of unit testing
is that it is automated, you write test code once and as the code
evolves if the new changes break the tests you will be notified. 

\section*{Getting Started \label{sec:Getting-Started}}

The example code provided at \href{file:./testingExample}{testingExample}
demonstrates the basics of creating your own unit testing. Many libraries
currently exist to help provide unit testing, some popular libraries
are CUnit and CppUnit. In our example, three functions have been created
to display the results of unit tests in the test.h and test.c. This
demonstrates the simplicity of unit testing statements.

In the testingExample, we have created two mains main.c and testMain.c
the first contains the functionality for running the programming as
normal. The second main file, testMain.c, contains functionality for
all the unit tests and summary of results. The concept of code-coverage
means that code you developed is covered by a unit testing or other
testing method. Full code-coverage indicates that you have unit-testing
for all or 90+\% of your code. Code coverage is measured in percentage
and the amount of code coverage depends on the application. If your
application is system-critical code coverage will consistent of a
high 95-99\%, less critical system will have lower code coverages.

Lets begin with writing a test, no matter what type of test you're
performing, your test should include the following information. This
example is from testMain.c:

\begin{lstlisting}[language=make,showstringspaces=false,tabsize=4,frame=TB]
Array * arr = createArray(-1);
printTestIntroInfo(testNumber++, "Testing invalid array size");
printf("Expected:%d\n", NULL);
printf("Received:%d\n", arr->arr == NULL);
testsPassed += printPassFail( arr->arr == NULL );
destroyArray(&arr);
\end{lstlisting}

This section of code, creates an Array structure with the size of
-1, this is an invalid size for the array, so our test evaluates how
the Array structure performs with invalid data sizes. After creating
the array, the first print statement indicates to the user what is
being tested. The next line prints what the expected result of the
test, followed by the actual results. On line 5, we continue summing
the number of tests that have passed and printed the result of the
test. Note, that the result of the test is a boolean result, we check
for equality between the expected result (NULL) and the actual result
found in arr-\textgreater{}arr. Finally, we clean up any memory used
for the test and that concludes a single unit test.

This is a single unit test for the creation of an Array, you will
likely need to include multiple tests to ensure higher quality and
coverage. For the array creation function, you should also test a
regular case, ie an array created with a valid size. Creating an array
with a size of -1 or 0 are considered edge cases, with size -1 the
input data is invalid, with a size of 0, it is valid however not useful.
Edge cases consider where your program is likely to break, based on
assumptions in your code or incorrect input. As an example the array
is created with a static size, we should test if what happens when
someone the array structure tries to access outside of that static
size.

The testMain.c includes a number of examples of unit tests, remember
that nearly every function of your abstract data type or container
should contain at least one unit test, this include creating and destroying
structures. When creating software it is occasional useful to consider
how you would test your code before actually writing, this is known
a test-driven development (TDD). In test-driven development, the developer
designs and implements the test before writing the code. The benefit
of this approach is minimizing the amount of code created and ensuring
each step is correct before progressing although it can be more time-
consuming.

\section*{Additional Information \label{sec:Additional-Information}}

The following links provide valuable additional introductory documentation
and examples. 
\begin{itemize}
\item \url{http://cunit.sourceforge.net/} 
\item \url{https://en.wikipedia.org/wiki/CppUnit} 
\item \url{http://agiledata.org/essays/tdd.html} 
\item \url{http://softwaretestingfundamentals.com/unit-testing/}
\end{itemize}

\end{document}
