%% LyX 2.2.1 created this file.  For more info, see http://www.lyx.org/.
%% Do not edit unless you really know what you are doing.
\documentclass[12pt,english]{article}
\renewcommand{\familydefault}{\sfdefault}
\usepackage{geometry}
\geometry{verbose,tmargin=1in,bmargin=1in,lmargin=1cm,rmargin=1in}
\usepackage{url}
\usepackage{amsmath}
\usepackage{amsthm}

\makeatletter
%%%%%%%%%%%%%%%%%%%%%%%%%%%%%% Textclass specific LaTeX commands.
\numberwithin{equation}{section}
\numberwithin{figure}{section}

%%%%%%%%%%%%%%%%%%%%%%%%%%%%%% User specified LaTeX commands.
\date{ }

\makeatother

\usepackage{babel}
\usepackage{listings}
\renewcommand{\lstlistingname}{Listing}

\begin{document}

\title{Review: CppCheck}
\maketitle

\section*{Overview}

CppCheck is a static analysis tool for C-based programming languages,
such as C and C++. Static analysis tools evaluate source code looking
for bugs beyond language specific syntax. This review introduces the
Cppcheck tool and how it can be used to evaluate and improve your
work.

\section*{Learning Outcomes \label{sec:Learning-Outcomes}}

Upon successful completion of this review, you should be able to:
\begin{itemize}
\item utilize the Cppcheck tool to evaluate source code
\item recognize the benefits and limitations of Cppcheck
\item maximize their testing efficiency by using a range of Cppcheck options
\end{itemize}

\section*{Introduction \label{sec:Introduction}}

Cppcheck is a command line software tool used for static analysis,
it is an open-source project that is available for download ( at\url{ https://sourceforge.net/p/Cppcheck/wiki/Home/}).
Static analysis tools perform similar operations to your compiler,
they evaluate your software code without running the code. A compiler
identifies language syntax errors, such as not following the rules
of the C-language, while static analysis tools like Cppcheck evaluate
the code for style and logical errors. Cppcheck will identify many
different types of errors in software code, some examples are out-of-bounds
on arrays, memory leaks, uninitialized, or unused code. Cppcheck as
a tool aims for 0 false positives, this means the tool errors on the
side of caution, only returning feedback with a high degree of certainty
of being an error. Cppcheck is a static analysis tool, this means
it does not run your code and is not able to identify most run-time
errors. 

\section*{Getting Started \label{sec:Getting-Started}}

Cppcheck is a command line tool that is installed on all university
computer science machines. If installed, entering Cppcheck into the
command line should display the Cppcheck manual page. The manual provides
an overview of the Cppcheck arguments and options. The code for this
review is available at (\#\#TODO: Link here\#\#). Here is an example
main.c file used to demonstrate different properties of Cppcheck.

\begin{lstlisting}[language=make,showstringspaces=false,tabsize=4,frame=TB]
/*  Purpose: This file purposely contains errors to be identified by Cppcheck.  
	File: main.c 
	University of Guelph, 2017.
*/
#include<stdio.h> 
#include<stdlib.h>
#include<string.h> 
#include<time.h>

/**This function does not and is not called.  
 *@pre No pre conditions  
 *@post No post conditions  
 *@param a void pointer that does nothing.
**/ 
void doNotRunMe(const void* vp) { return; }

int main(int argc, char ** argv) {   
	 // The following code purposely contains errors    
	// Ensure you can identify the same errors with gcc or Cppcheck.
    int * leakyPtr = malloc(sizeof(int) );  
	*leakyPtr = 5;
    printf("The value is : %d ", leakyPtr);
    printf("The value is : %d ", *leakyPtr);

    char * chptr = NULL;
    if( *leakyPtr == 5) {
		chptr = malloc(sizeof(char));
        chptr = 'b';
	}
    printf("ChPtr=%c", *chptr);
    return 0;
} 
\end{lstlisting}

When you run Cppcheck on this single file type, you should see feedback
similar the following output:
\begin{lstlisting}[language=make,showstringspaces=false,tabsize=4,frame=TB]
> cppcheck main.c

//Output
Checking main.c... 
[main.c:39]: (error) Memory leak: chptr 
[main.c:46]: (error) Memory leak: leakyPtr
\end{lstlisting}

Running Cppcheck on main.c has identified two errors both memory leaks
on the two pointers. The output is formatted as {[} filename:line
number{]} (error type) Description : Additional information ( such
as variable names ). Cppcheck is capable of identifying basic memory
leaks, however, this is not the tools main function and tools such
as Valgrind are much more reliable for identifying memory leaks. The
output format can be modified or even changed to XML(via the \textendash xml
flag) if desired but the standard output format should be sufficient.

\subsection*{Flags\label{subsec:Variables}}

This section discusses the most commonly used Cppcheck flags, these
are all available for review in the Cppcheck manual. 

\subsubsection*{Enable}

The enable flag specifies which types of errors will be presented
to the user, by default most errors are not enable, to quickly enable
all errors type

\begin{lstlisting}[language=make,showstringspaces=false,tabsize=4,frame=TB]
cppcheck --enable=all main.c 

//Output
[main.c:31]: (warning) %d in format string (no. 1) requires
			'int' but the argument type is 'int *'. 

[main.c:42]: (warning) Possible null pointer dereference: chptr
[main.c:38] ->
[main.c:39]: (style) Variable 'chptr' is reassigned
			a value before the old one has been used. 

[main.c:39]: (error) Memory leak: chptr 
[main.c:46]: (error) Memory leak: leakyPtr 
[main.c:19]: (style) The function 'doNotRunMe' is never used. 

(information) Cppcheck cannot find all the include files
			(use --check-config for details)
\end{lstlisting}

When all Cppcheck errors are enabled the output includes many other
types of warnings, suggestions, and recommendations. From the output,
we observe two additional warnings, the first indicates that the print
function on Line 31 is actually passing the pointer not the integer
value, this is indicated as a warning because perhaps the developer
wanted to print the pointer's address not the value. The second error
on Line 42 indicates a potential error that could occur if leakyPtr
was no longer set to 5 then the print statement on Line 42 could potentially
be null and we would be dereferencing a null pointer, which would
cause a crash. Cppcheck also identifies that on Line 19, we have a
function that is never used, this is marked as a style error because
perhaps this function is used elsewhere or will be used later. Finally,
Cppcheck also provides information that not all included files can
be found, this can occasionally occur when the standard libraries
are includes but the path is not provided. If we run Cppcheck again
with the suggested flag \textendash check-config, it will indicate
the path is not set for proper standard library headers. This information
can be ignored or omitted from the results using the suppress flag,
ie ( Cppcheck \textendash suppress=missingIncludeSystem). The enable
flag supports other more specific types of errors beyond all if you
want to focus on a particular type of error you can use the following
options warning, style, performance, portability, information, unusedFunction,
or missingInclude. As an example (Cppcheck \textendash enable=information
main.c), would display only information about the source code.

\subsubsection*{Including Header files}

To include your header files so that you don't receive information
warnings about your non-included header files add the -I flag and
your include directory.

\begin{lstlisting}[language=make,showstringspaces=false,tabsize=4,frame=TB]
cppcheck --enable=all -I ./include/ main.c 

//Output
[main.c:31]: (warning) %d in format string (no. 1) requires 'int'
			but the argument type is 'int *'. 

[main.c:42]: (warning) Possible null pointer dereference: chptr 
[main.c:38] ->
[main.c:39]: (style) Variable 'chptr' is reassigned
			a value before the old one has been used. 

[main.c:39]: (error) Memory leak: chptr 
[main.c:46]: (error) Memory leak: leakyPtr 
[main.c:19]: (style) The function 'doNotRunMe' is never used. 
\end{lstlisting}


\subsubsection*{Errors and Suppressing}

The suppress flag can be used to ignore errors warnings you don't
wish to receive if you run Cppcheck with missingIncludeSystem and
memleak suppressed you will receive the same output as before except
without the memory leaks and include information. To find the tags
name of errors you want to suppress use the flag: \textendash errorlist.
Errorlist will display all errors Cppcheck searches for in XML format.

\begin{lstlisting}[language=make,showstringspaces=false,tabsize=4,frame=TB]
cppcheck --enable=all --suppress=missingIncludeSystem --suppress=memleak main.c 

//Output
[main.c:31]: (warning) %d in format string (no. 1) requires 'int'
			but the argument type is 'int *'.

[main.c:42]: (warning) Possible null pointer dereference: chptr 
[main.c:38] -> [main.c:39]: (style) Variable 'chptr' is reassigned
			a value before the old one has been used.

[main.c:19]: (style) The function 'doNotRunMe' is never used.
\end{lstlisting}


\subsubsection*{Language and Standards}

Cppcheck can evaluate your source code based on different C and C++
language settings or standards. Cppcheck will naturally access your
code to identify a language but if you want to specify the language,
use the language (\textendash language flag). C and C++ standards
dictate the progression of the language, for instance, c99 indicates
the C language standard as of 1999, as the language evolves new standards
are set to support development needs. The difference between C/C++
languages standard could be the implementation of new features such
as the inclusion of strings, booleans or newer features. The Cppcheck
defaults to the newest standard so if you want to use an older version,
you will have to specify it with the standard flag. The example demonstrates
both language and standard flags, set to C and C99 standard. Notice
that the example also doesn't specify a specific file but now includes
any files in the src directory. Depending on the portability of your
code to older machines, setting the C/C++ language standard to an
older version might not include any new errors or warnings.

\begin{lstlisting}[language=make,showstringspaces=false,tabsize=4,frame=TB]
cppcheck --enable=all --language=c --std=c99 ./src/* 

//Output
[src/main.c:33]: (warning) %d in format string (no. 1) requires 'int'
				but the argument type is 'int *'.

[src/main.c:47]: (warning) Possible null pointer dereference: chptr 
[src/main.c:40] ->
[src/main.c:41]: (style) Variable 'chptr' is reassigned
				a value before the old one has been used. 

[src/main.c:41]: (error) Memory leak: chptr 
[src/main.c:51]: (error) Memory leak: leakyPtr
[src/main.c:21]: (style) The function 'doNotRunMe' is never used. 

(information) Cppcheck cannot find all the include files
			(use --check-config for details)
\end{lstlisting}


\subsubsection*{Inconclusive}

Cppcheck as a tool aims for 0 false positives, this means that occasionally
Cppcheck might be aware of potential errors but is not confident enough
to display them as an error. The inconclusive flag allows observation
of all potential warnings and errors even if Cppcheck is uncertain
of the statement. The inconclusive flag can be useful for identifying
other potential errors, but inconclusive errors should be evaluated
with a higher level of scrutiny because they might not be errors at
all. To include inconclusive errors use the \textendash inconclusive
flag.

\begin{lstlisting}[language=make,showstringspaces=false,tabsize=4,frame=TB]
cppcheck --enable=all --language=c --std=c99 --inconclusive -I ./include/ ./src/*

//Output
[src/main.c:33]: (warning) %d in format string (no. 1) requires 'int'
				but the argument type is 'int *'.

[src/main.c:47]: (warning) Possible null pointer dereference: chptr 
[src/main.c:40] ->
[src/main.c:41]: (style) Variable 'chptr' is reassigned
				a value before the old one has been used. 

[src/main.c:41]: (error) Memory leak: chptr 
[src/main.c:51]: (error) Memory leak: leakyPtr
[src/main.c:21]: (style) The function 'doNotRunMe' is never used. 

(information) Cppcheck cannot find all the include files
			(use --check-config for details)
\end{lstlisting}

Note the output remains the same with the use of the inconclusive
flag, as most of the errors in the main.c are straight forward.

\section*{Additional Information \label{sec:Additional-Information}}

Cppcheck is a useful tool for identifying potential issues with your
project for additional reference please see the following links:
\begin{itemize}
\item Main Website ( \url{http://cppcheck.sourceforge.net/} )
\item Official Manual ( \url{http://Cppcheck.sourceforge.net/manual.pdf}
)
\item Useful Manual ( \url{http://www.cs.kent.edu/~rothstei/spring_12/secprognotes/Cppcheck_manual.pdf })
\end{itemize}

\end{document}
