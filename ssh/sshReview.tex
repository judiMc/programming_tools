%% LyX 2.2.2 created this file.  For more info, see http://www.lyx.org/.
%% Do not edit unless you really know what you are doing.
\documentclass[12pt,english]{article}
\renewcommand{\familydefault}{\sfdefault}
\usepackage{geometry}
\geometry{verbose,tmargin=1in,bmargin=1in,lmargin=1cm,rmargin=1in}
\usepackage{url}
\usepackage{amsmath}
\usepackage{amsthm}

\makeatletter
%%%%%%%%%%%%%%%%%%%%%%%%%%%%%% Textclass specific LaTeX commands.
\numberwithin{equation}{section}
\numberwithin{figure}{section}

%%%%%%%%%%%%%%%%%%%%%%%%%%%%%% User specified LaTeX commands.
\date{ }

\makeatother

\usepackage{babel}
\usepackage{listings}
\renewcommand{\lstlistingname}{Listing}

\begin{document}

\title{Review: SSH}
\maketitle

\section*{Overview}

SSH ( Secure Shell ) is a program used to securely connecting to a
remote server. SSH allows you connect to the university server and
use the programming tools already established. This review covers
a basic introduction to the topic, including a ``Getting Started''
example, and some common best practice. For this review, it is suggested
to read the additional support section before following any of the
steps in the getting started, as the AnyConnect software can greatly
ease connection to university servers for the first time.

\section*{Learning Outcomes \label{sec:Learning-Outcomes}}

Upon successful completion of this review, you should be able to:
\begin{itemize}
\item recognize the benefits of SSH 
\item use SSH to connect to a remote server
\item manage and test your software project remotely
\end{itemize}

\section*{Introduction \label{sec:Introduction}}

SSH is terminal program that allows for secure encrypted connection
between a host (your computer) and remote server. Occasionally, you
may need to remote connect to the university server to test your code,
if you are unable to be physically present at the university. SSH
communicates over using the SSH protocol to the remote server. Once
a secure connection has been established via SSH you will have complete
access to your account, as if you were physically present. Within
a Linux or Mac environment you can use the command SSH in the command
line, if you're on windows the currently recommended application to
connect is via PUTTY, instructions are available here (\url{https://www.uoguelph.ca/ccs/software/supported-products/putty})

\subsection*{SSH Public or Private Keys}

SSH authenticates users using either a password or an SSH key. SSH
uses a set of public and private keys, the public key is shared to
users of the system, while the private keys are protected and restricted.
The remote system keeps a log of all authorized public keys that it
has generated and in combination with the private key allows entry.
To generate a key you can type : ssh-keygen -t rsa -C yourusername@linux.socs.uoguelph.ca 

A key will be generated with the passphrase you provide and it will
be stored in your home directory in the folder .ssh by default named
id\_rsa.pub. The file id\_rsa.pub will contain the key you will need
to pass to the linux.socs.uoguelph.ca when your first login.

\section*{Getting Started \label{sec:Getting-Started}}

If you type the SSH command into the terminal using: 
\begin{lstlisting}[language=make,tabsize=4,frame=TB]
ssh yourUserName@general.uoguelph.ca
\end{lstlisting}

SSH will request a password, if you don't setup an SSH key, you will
have to enter that password every time you login. If you setup an
SSH key, you will be automatically recognized and provided access.
A useful tutorial on installing SSH keys is provided by the university
at \url{https://www.uoguelph.ca/ccs/cwc/installing-ssh-keys}. From
general.uoguelph.ca, you will need to log into linux.socs.uoguelph.ca
to establish the connection between the authorization keys. In linux.socs.uoguelph.ca,
change directories to .ssh and add a file called authorized\_keys
and paste the generated key to the bottom of the file.

The next time you login using the terminal or putty you should be
able to directly login to ssh to linux.socs.uoguelph.ca

\section*{Additional Support\label{sec:Examples} }

Cisco AnyConnect VPN is software designed to improve the ease of connecting
and managing remote access. It is highly recommended if you use Mac
or Windows to use AnyConnect first before SSH into your linux.socs.uoguelph.ca.
Instructions for installing and getting started with AnyConnect is
available at \url{https://www.uoguelph.ca/ccs/anyconnect-vpn-user-guide}.
Once connected to the AnyConnect VPN you will not need to connect
to the general.uoguelph.ca server, you can connect directly to the
linux.socs server.

\section*{Additional Information \label{sec:Additional-Information}}
\begin{itemize}
\item \url{https://www.digitalocean.com/community/tutorials/ssh-essentials-working-with-ssh-servers-clients-and-keys}
\item \url{https://www.uoguelph.ca/ccs/cwc/installing-ssh-keys}
\item \url{https://www.uoguelph.ca/ccs/internet-phones/other-ways-get-connected/shell-access}
\item \url{https://www.uoguelph.ca/ccs/cwc/sftp-tunneling-over-ssh-putty}
\end{itemize}

\end{document}
