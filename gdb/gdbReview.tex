%% LyX 2.2.2 created this file.  For more info, see http://www.lyx.org/.
%% Do not edit unless you really know what you are doing.
\documentclass[12pt,english]{article}
\renewcommand{\familydefault}{\sfdefault}
\usepackage{geometry}
\geometry{verbose,tmargin=1in,bmargin=1in,lmargin=1cm,rmargin=1in}
\usepackage{url}
\usepackage{amsmath}
\usepackage{amsthm}

\makeatletter
%%%%%%%%%%%%%%%%%%%%%%%%%%%%%% Textclass specific LaTeX commands.
\numberwithin{equation}{section}
\numberwithin{figure}{section}

%%%%%%%%%%%%%%%%%%%%%%%%%%%%%% User specified LaTeX commands.
\date{ }

\makeatother

\usepackage{babel}
\usepackage{listings}
\renewcommand{\lstlistingname}{Listing}

\begin{document}

\title{Review: GDB}
\maketitle

\section*{Overview}

GDB is the GNU debugger, it is a debugger for several languages including
C and C++. Understanding how to use different debugging tools can
greatly reduce the amount of time you spend fixing or code. We will
review an introduction to GDB, including a ``Getting Started'' example,
and some common best practice.

\section*{Learning Outcomes \label{sec:Learning-Outcomes}}

Upon successful completion of this review, you should be able to:
\begin{itemize}
\item recognize the benefits of the GNU Debugger
\item comprehend the different GDB options
\item apply GDB options to inspects and evaluate your code
\end{itemize}

\section*{Introduction \label{sec:Introduction}}

A debugger is a computer program that aids in evaluating and testing
your source code. Debuggers allow you to step through your program
line by line, this helps observe the state of your code and identify
bugs. A debugger provide a layer of protection for your system when
running software. The potential for errors in your software code to
modify or damage a system are possible, although very unlikely, a
debugger helps protect your system.

\section*{Getting Started \label{sec:Getting-Started}}

Debugging your code to ensure it is working properly can be a frustration
and time-consuming task. Developing practical experience using a debugger
and other evaluation tools can improve your ability to quickly find
errors and ensure a higher level of quality code. Please see the review
material on unit testing for developing small automated tests.

A debugger will help you find instances where your program has crashed,
one of the most common cases are segmentation faults in which the
software has tried to use too much memory or access unavailable system
memory.

GBD is integrated with GCC if you would like an additional resource
on GCC please see the review materials. By adding the -g command line
parameter to GCC compiler call you enable debugging support. For instance

\begin{lstlisting}[language=make,showstringspaces=false,tabsize=4,frame=TB]
gcc -Wall -g main.c -o myProgram
\end{lstlisting}

By adding debugging support to your program you can know use GDB.
Debugging support adds additional information to your program to help
trace the steps, this includes leaving some information in a human
readable form.

GDB has a user interface that you can explore by simply type gdb in
the command line. Once in gdb interface, you can type help to explore
other options. To begin using GDB with your program simply use the
following command.

\begin{lstlisting}[language=make,showstringspaces=false,tabsize=4,frame=TB]
gdb myProgram
\end{lstlisting}

This begins a GDB session with myProgram, if you type run, your program
will be run in the safety of the GDB session.

\section*{Examples \label{sec:Examples} }

Let's examine a more practical example, here is an example of running
a small program with gdb run, here is the main of this example:

\begin{lstlisting}[language=make,showstringspaces=false,tabsize=4,frame=TB]
int * p = NULL;
printf("This main is about to segfault\n");
printf("%d",*p); 
\end{lstlisting}

These three lines of code create a segmentation fault. The code contains
a pointer that is set to NULL and when accessed in the printf statement
it tries to convert the value to a decimal. If you run the example
in GDB, the output will be:

\begin{lstlisting}[language=make,showstringspaces=false,tabsize=4,frame=TB]
Starting program: mainTest  
This main is about to segfault
Program received signal SIGSEGV, Segmentation fault. 
0x000000000040058b in main (argc=1, argv=0x7fffffffdd98) at main.c:27 
27 printf("%d",*p); 
\end{lstlisting}

The output first demonstrates that mainTest was run, followed by our
warning statement that we expected the program to segfault. The 3rd
line of the output is the GDB output indicating that the program has
segfaulted. The fourth line of output is a hexadecimal number of the
system memory address of your the program, followed by the file (main.c)
and the line number 27. The last line of the output indicates the
line of code the segmentation fault occurred on. It's important to
note that the line a code that crashes is not necessarily the cause
of the crash, the crash is often caused by a logical error earlier
in the program. For instance, in our code, the printf(``\%d'', {*}p)
is valid code that only produces a segmentation fault if the pointer
is invalidate or NULL. The logical error in this code is not providing
an integer value for the pointer p, either in the first step or before
the print statement. Alternatively, this error could be avoided by
prechecking that {*}p does not equal NULL before printing.

By typing run without using any other features of the debugger the
program runs until a crash, completion, or user input is required.
In our instance, the debugger indicates that an error occurs at line
27 if we wanted to halt the program before the error to view the state
of the program we can add a breakpoint. A breakpoint is exactly what
is sounds like, a stopping point in which your program will pause
execution. To create a breakpoint in the main.c file at line 20 in
GDB interface type:

\begin{lstlisting}[language=make,showstringspaces=false,tabsize=4,frame=TB]
break main.c:20
\end{lstlisting}

You can add as many breakpoints as needed. Now in GDB when you type
run, your program will run up to line 20 then stop until further instruction
with the following output:

\begin{lstlisting}[language=make,showstringspaces=false,tabsize=4,frame=TB]
Breakpoint 1, main (argc=1, argv=0x7fffffffdd98) at main.c:24 
24		printf("%d",*p); 
\end{lstlisting}

Line 20 is comment thus GDB breaks at the next instruction which occurs
on line 24. Once stopped at a breakpoint you can view the value of
the p variable by typing:

\begin{lstlisting}[language=make,showstringspaces=false,tabsize=4,frame=TB]
// Print the value of p
print {p}

// Print the type and value of p
print p
\end{lstlisting}

Once you've reached a breakpoint, you might want to navigate slowly
to the next line of code to inspect the state of your program in greater
detail.

\begin{lstlisting}[language=make,showstringspaces=false,tabsize=4,frame=TB]
// type next to execute the next line of code
next

// step is similar to next except that it will enter into a function.
step

// type continue to keep running the program past a breakpoint
continue;
\end{lstlisting}


\section*{Additional Information \label{sec:Additional-Information}}
\begin{itemize}
\item \url{https://www.cs.umd.edu/~srhuang/teaching/cmsc212/gdb-tutorial-handout.pdf}
\item \url{http://www.thegeekstuff.com/2010/03/debug-c-program-using-gdb/}
\item \url{http://web.eecs.umich.edu/~sugih/pointers/gdbQS.html}
\end{itemize}

\end{document}
