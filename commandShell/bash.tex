%% LyX 2.2.2 created this file.  For more info, see http://www.lyx.org/.
%% Do not edit unless you really know what you are doing.
\documentclass[12pt,english]{article}
\renewcommand{\familydefault}{\sfdefault}
\usepackage[latin9]{inputenc}
\usepackage{geometry}
\geometry{verbose,tmargin=1in,bmargin=1in,lmargin=1cm,rmargin=1in}
\usepackage{url}
\usepackage{amsmath}

\makeatletter
%%%%%%%%%%%%%%%%%%%%%%%%%%%%%% User specified LaTeX commands.
\date{ }

\usepackage{babel}
\usepackage{listings}
\renewcommand{\lstlistingname}{Listing}

\usepackage{babel}
\usepackage{listings}
\renewcommand{\lstlistingname}{Listing}

\makeatother

\usepackage{babel}
\usepackage{listings}
\renewcommand{\lstlistingname}{Listing}

\begin{document}

\title{Review: Command Shell}
\maketitle

\section*{Overview}

The command shell, command line, or shell is an user interface which
interacts directly with the operating system. Mastering fundamentals
of the command shell will be integral in developing your labs and
assignments throughout the course. We will discuss the most common
commands, identify the most common types of command shells, discuss
how to call system commands in a ``Getting Started'' example, and
additional resources.

\section*{Learning Outcomes \label{sec:Learning-Outcomes}}

Upon successful completion of this review, you should be able to: 
\begin{itemize}
\item comprehend the difference between CLI and GUI
\item be aware of alternative types of command shell
\item identify the 10 most common commands in bash
\item call shell commands from your C program
\end{itemize}

\section*{Introduction \label{sec:Introduction}}

The command shell is the original user interface for most operating
systems, it allows users to type commands to start programs. While
many of us have grown accustomed to graphical user interfaces, the
command-line interface (CLI) is still a very powerful method of
interacting with the operating system. The command shell is a program
that takes user input, finds, then runs the appropriate program with
the arguments provided. On most Linux operating systems, the default
shell is usually sh (Bourne Shell) or bash (Bourne Again Shell).
If you a start a terminal in Linux, you'll likely be using bash. Different
shells have a number of different preferences and commands. Examples
of different shells are zsh, ksh, and tcsh. In addition, there are
multiple terminal-emulators that can improve usability such xterm,
terminator, and terminal.

As bash is the default command shell, that is where we will focus
our attention. The importance of working from the shell is understanding
how the filesystem is setup. In our GUI, files and folders have a
visual representation that mimicks the underinglying filesystem. Instead
of clicking on folder to enter the folder, you must use the command
(\texttt{cd}) to change directories.

\section*{Getting Started \label{sec:Getting-Started}}

The bash command line provides a powerful method of running software,
one of the key advantages is being able to generate scripts that automate
common commands process.

\subsubsection*{Getting Help}

Most bash commands will provide a helpful description of the program,
the list of possible arguments, and examples. If a bash command or
program has additional information, it will usually be available via
one of these standard methods.

\begin{lstlisting}[language=make,showstringspaces=false,tabsize=4,frame=TB]
# provide manual for ls command
man ls

# provide either -h or --help as an argument to a program for help
someCommand -h
\end{lstlisting}


\subsubsection*{Navigation}

Navigation has several shortcuts to refer to specific locations. A
'.' refers to the current directory, while '..' refers to the parent
directory. The '\textasciitilde{}' symbol refers to your home directory.
The '/' symbol refers to the root directory of the computer. To begin
navigating your bash shell, several commands are useful

\subsubsection*{List}

To list the contents of a directory, call the command \texttt{ls} followed
by a path to a directory

\begin{lstlisting}[language=make,showstringspaces=false,tabsize=4,frame=TB]
# To list what is the current directory
ls

# List some sub-directory
ls subDirectory/AnotherDir/

# List all hidden files and information in the current directory
ls -al
\end{lstlisting}


\subsubsection*{Change Directory}
To move about your filesystem, use the \texttt{cd} command, followed by a
filepath.

\begin{lstlisting}[language=make,showstringspaces=false,tabsize=4,frame=TB]
# Change directory from your current location
cd subDirectory/anotherSubDirectory
# or
cd ./subDirectory/anotherSubDirectory

# Change directory from your home directory.
cd ~/someDirectory/

# Change directory from root directory
cd /var/user/yourUserName

# Change directories back to home
cd
\end{lstlisting}


\subsubsection*{Print Working Directory}

If you would like to view your current directory use the \texttt{pwd} command

\begin{lstlisting}[language=make,showstringspaces=false,tabsize=4,frame=TB]
pwd
\end{lstlisting}


\subsubsection*{Clear screen}

To clear the screen of any output use the \texttt{clear} command

\begin{lstlisting}[language=make,showstringspaces=false,tabsize=4,frame=TB]
clear
\end{lstlisting}


\subsubsection*{History}

To view the history of commands called from your shell use the
\texttt{history} command.
This will print recent command calls to \texttt{stdout}.

\begin{lstlisting}[language=make,showstringspaces=false,tabsize=4,frame=TB]
history
\end{lstlisting}


\subsubsection*{Previewing Files}

To view the contents of a file there are several functions which are
helpful. Alternatively, text-editors such as \texttt{nano},
\texttt{vim} or \texttt{emacs} exist if you which to view and edit file
contents.

To view just the beginning of a file use the \texttt{head} command

\begin{lstlisting}[language=make,showstringspaces=false,tabsize=4,frame=TB]
# View top contents of a file
head someFile.txt

# To view the first 20 lines of a file, use the -n argument.
head -n 20 someFile.txt
\end{lstlisting}

The \texttt{tail} command allows you to view the last section of content in
a file, this is useful for reading errors or the most recent output
from a program.

\begin{lstlisting}[language=make,showstringspaces=false,tabsize=4,frame=TB]
# View top contents of a file
tail someFile.txt

# To view the first 20 lines of a file, use the 'n' argument.
tail -n 20 someFile.txt
\end{lstlisting}

Finally, if you which to view the full file contents in manageable
sections, the \texttt{more} command breaks the file into sections to be
viewed, allowing one to scroll down the output at a user-controlled pace.

\begin{lstlisting}[language=make,showstringspaces=false,tabsize=4,frame=TB]
# View the full contents of a file in sections
more someFile.txt
\end{lstlisting}

The \texttt{less} command provides similar functionality to \texttt{more},
but allows one to scroll up and down.


\subsection*{Piping}
Another advantage of command line software is being able to quickly share
data between programs via a process called 'piping'.

Programs that share information via \texttt{stdout} and \texttt{stdin} may
be chained together with ``pipes" represented by the $|$ character.

Programs that are connected by pipes have their inputs and outputs flow
from left to right. In the following example, we will pipe the output of
the \texttt{history} command into the program \texttt{more}. This will make
it easier read your shell command execution history.

\begin{lstlisting}[language=make,showstringspaces=false,tabsize=4,frame=TB]
# View the contents of history in sections.
history | more
\end{lstlisting}

Piping can be even much more powerful, see the following example which
chains together some standard Unix utilities to produce 10 unique 12
character alphanumeric strings.

Note that/dev/urandom is a ``device" that provides an interface to the
kernel's random number generator.

\begin{lstlisting}[language=make,showstringspaces=false,tabsize=4,frame=TB]
# cat reads from /dev/urandom, printing ASCII values to stdout
# tr reads characters from stdin and filters out anything that is not a
# letter or number
# fold -w 12 is used to force line wraps every 12 characters
# finally, head -n 10 grabs only the first 10 lines from stdin and prints them
cat /dev/urandom | tr -dc "a-zA-Z0-9" | fold -w 12 | head -n 10
\end{lstlisting}

We will delve into some more examples and utilities in the following sections.




\section*{Examples in C \label{sec:Examples in C}}

Shell commands can be called directly from your C program.
Consider the program provided in commandShell.c. This program takes
multiple arguments that are paths to directorys and lists the contents
of each directory.

\begin{lstlisting}[language=make,showstringspaces=false,tabsize=4,frame=TB]
#include<stdio.h>
#include<stdlib.h>
#include<string.h>

int main( int argc, char ** argv )
{
    for( int i = 1; i < argc; i++) {
        char path[256];
		// This function combines argument with the list command.
		// ex) ls someDirectory
        snprintf(path, sizeof path, "%s%s", "ls  ", argv[i]);

		// This function calls the operating system with the string
		// provided in this case a list function
        system(path);
    }
}
\end{lstlisting}


\section*{Additional Information \label{sec:Additional-Information}}
\begin{itemize}
\item \url{http://linuxcommand.org/index.php}
\item \url{https://www.gnu.org/software/bash/}
\item \url{http://www.informit.com/blogs/blog.aspx?uk=The-10-Most-Important-Linux-Commands}
\end{itemize}

\end{document}
