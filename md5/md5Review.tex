%% LyX 2.2.2 created this file.  For more info, see http://www.lyx.org/.
%% Do not edit unless you really know what you are doing.
\documentclass[12pt,english]{article}
\renewcommand{\familydefault}{\sfdefault}
\usepackage{geometry}
\geometry{verbose,tmargin=1in,bmargin=1in,lmargin=1cm,rmargin=1in}
\usepackage{url}
\usepackage{amsmath}
\usepackage{amsthm}

\makeatletter
%%%%%%%%%%%%%%%%%%%%%%%%%%%%%% Textclass specific LaTeX commands.
\numberwithin{equation}{section}
\numberwithin{figure}{section}

%%%%%%%%%%%%%%%%%%%%%%%%%%%%%% User specified LaTeX commands.
\date{ }

\makeatother

\usepackage{babel}
\usepackage{listings}
\renewcommand{\lstlistingname}{Listing}

\begin{document}

\title{Review: MD5}
\maketitle

\section*{Overview}

MD5 is an algorithm used to create a 128-bit hash value. Initially
MD5 was created for security purposes but it no longer is the most secure way to encrypt
information.  However, it is an accessible introduction to using encryption.  This review covers a basic
introduction to the topic, including a ``Getting Started'' example,
and some common applications.

\section*{Learning Outcomes \label{sec:Learning-Outcomes}}

Upon successful completion of this review, you should be able to:
\begin{itemize}
\item recognize the applications of MD5
\item generate MD5 hashes
\item use MD5 to validate data
\end{itemize}

\section*{Introduction \label{sec:Introduction}}

MD5 was an algorithm originally created for security, however the
algorithm has several vulnerabilities that make it no longer suitable
for protection. Now MD5 is mainly used for data validation and integrity,
via checksums. MD5 is a hash function that hash values of 128 bits.
MD5 is mainly a method of quickly evaluating transmitted data, they
check the data by using a hash function that creates a summation of
the binary data. Checksums do not 100\% guarantee the integrity of
the transmitted data is correct Although, it's very unlikely that
the hash value of corrupted data will match the hash value of a successful
transferred file it can occur.

MD5 checksums known as md5sum, occurs on many servers, as an initial
step in validation.

\section*{Getting Started \label{sec:Getting-Started}}

To generate a md5sum for your text or data, in the terminal of a Linux
system, type:
\begin{center}
\begin{lstlisting}[language=make,showstringspaces=false,tabsize=4,frame=TB]
echo -n "Your Text" | md5sum
\end{lstlisting}
\par\end{center}

This creates a md5 checksum of whatever text you've written, if you
change the text a different hash value will be created. Echo -n is
an option that does not excludes the additional line break included
by echo. Then your text is piped into md5sum to create a hash value.
The expected MD5sum result should be provided to server when downloading
and a match can be validated once transfer is completed.

Creating a hash for a number of files to be transferred can be done
as follows:
\begin{center}
\begin{lstlisting}[language=make,showstringspaces=false,tabsize=4,frame=TB]
md5sum filetohashA.txt filetohashB.txt filetohashC.txt > hash.md5
\end{lstlisting}
\par\end{center}

If you view the hash.md5 file you will see a similar 128-bit hash
value for each file. If you want to check the md5sum are correct,
use the -c option.
\begin{center}
\begin{lstlisting}[language=make,showstringspaces=false,tabsize=4,frame=TB]
md5sum -c hash.md5
\end{lstlisting}
\par\end{center}

\section*{Additional Information \label{sec:Additional-Information}}

The following links can provide useful additional information on MD5:

\url{http://www.makeuseof.com/tag/md5-hash-stuff-means-technology-explained/}

\url{https://en.wikipedia.org/wiki/MD5}

\url{https://www.go4expert.com/articles/md5-tutorial-t319/}
\end{document}
