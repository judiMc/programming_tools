%% LyX 2.2.1 created this file.  For more info, see http://www.lyx.org/.
%% Do not edit unless you really know what you are doing.
\documentclass[12pt,english]{article}
\renewcommand{\familydefault}{\sfdefault}
\usepackage{geometry}
\geometry{verbose,tmargin=1in,bmargin=1in,lmargin=1cm,rmargin=1in}
\usepackage{babel}
\usepackage{url}
\usepackage{amsmath}
\usepackage{amsthm}
\usepackage[unicode=true]
 {hyperref}

\makeatletter
%%%%%%%%%%%%%%%%%%%%%%%%%%%%%% Textclass specific LaTeX commands.
\numberwithin{equation}{section}
\numberwithin{figure}{section}

%%%%%%%%%%%%%%%%%%%%%%%%%%%%%% User specified LaTeX commands.
\date{ }

\makeatother

\usepackage{listings}
\renewcommand{\lstlistingname}{Listing}

\begin{document}

\title{Review: Command Line Arguments}
\maketitle

\section*{Overview}

Command-line arguments allow a user or another program to specify
information to be passed into a program. This allows the user to customize
the behavior of the command-line run programs. This review will provide
a basic introduction to the topic, including a \textquotedblleft Getting
Started\textquotedblright{} example and some common best practice.

\section*{Learning Outcomes \label{sec:Learning-Outcomes}}

Upon successful completion of this review, you should be able to:
\begin{itemize}
\item recognize the benefits of command-line arguments
\item validate user input command-line arguments
\item create validate program which accept user provided arguments.
\end{itemize}

\section*{Introduction \label{sec:Introduction}}

Command-line arguments are pieces of information based on a program
before running your program. Each argument provides either data to
the program or information about how to customize behavior such output.
The standard for command line arguments is to use either a single
dash '-' followed by a single letter or double dash '\textendash '
followed by a word. A simple example of command line arguments in
use is in your terminal type, using the list function 'ls'. The list
command will list all the files in the directory if provided the command
line argument ``-a or -{}-all\textquotedblright , sometimes referred
to as flags, a list will provide all files in the directory including
hidden files.
\begin{center}
\begin{lstlisting}[language=make,showstringspaces=false,tabsize=4,frame=TB]
//Note: These two produce the same output.
ls -a
ls --all

//This produces a different result, but is the same program.
ls

// To see the full list of command line arguments for ls
man ls
\end{lstlisting}
\par\end{center}

Most professional software that uses a command line interface follows
this similar command line argument pattern. If you which to see the
full list of available arguments for a program you can retrieve the
manual by typing. As many command line arguments as needed can be
added to a single command for instance if we extend the original ls
example to add the flag ``-l'' after the ``-a'', the output would
display more information about each file. The remainder of this review
provides instruction for creating and managing your own command line
arguments.

\section*{Getting Started \label{sec:Getting-Started}}

From the introduction, we learned that command line arguments are
easy to use to change the behavior of our application. The arguments
follow the name of the program that we will be running and can be
used in conjunction to create different program behaviors and output.

Each C/C++ program contains the declaration for command line parameters
in the main function. The main function contains two arguments, the
first, argc (from argument count) is the number of command-line arguments
that where presented at the start of the program. The second, argv
( from argument variables), is array of strings containing the information
presented to the program as arguments.

\begin{lstlisting}[language=make,tabsize=4,frame=TB]
int main( int argc, char ** argv ) 
\end{lstlisting}

It is important to note that argc is always at least 1, the first
argument which is the name of your program is automatically passed-in
and stored in argv{[}0{]}. Thus if you provide additional command-line
parameters, they will begin at argv{[}1{]}. Here is an example call
for the code below, ``yourProgramName'' is what get stored in argv{[}0{]},
argv{[}1{]} would store ``someFile.txt'', argv{[}2{]} would have
50.

\begin{lstlisting}[language=make,tabsize=4,frame=TB]
./yourProgramName someFile.txt 50
\end{lstlisting}


\section*{Example \label{sec:Examples} }

Let's examine a more practical example, here is an example of the
main function that is available at (\href{file:./code/main.c}{Command Line Arguments main.c}).
The main is basically empty except that it checks for command line
arguments. Argc is essential to ensure the proper number of variables
exists, all of the data is stored in argv.

\begin{lstlisting}[language=make,showstringspaces=false,tabsize=4,frame=TB]
/**
 Description: Demonstrates basic usage of command line arguments
 University of Guelph, 2017
**/

#include<stdlib.h>
#include<stdio.h>

int main( int argc, char ** argv ) 
{
	/* Argc is the argument counter
	   It indicates the number of passed in parameters.

	   Argv is the actual parameters being passed in.	

	   Note that the first argument (argv[0]) is the program name.	
	*/

	if( argc == 1 ) {
		printf("Describe what your program does here\n");
		printf("What arguments does your program require\n");
		return 0;
	}

	if( argc > 1 && argc < 4) {
		// The first parameter is the program being run 
		char * programName = argv[0];
		printf("Here is what you ran: %s\n", programName );
		
		// Your first parameter is a filename
		char* filename = argv[1];
	
		// The second parameter is an integer
		int someNumber = atoi(argv[2]);

		printf("InputFile: %s\n", filename);
		printf("Number: %d\n", someNumber);
	}
	else {
		// You should check to make sure you get all the intended arguments.
		// More args could  harmless but could also be a bad sign.
		printf("An invalid number of arguments. Please see the description");
	}	
	return 0;
}
\end{lstlisting}

This example demonstrates three main components in the main, the first
being that your program should have a default behavior if it expects
command-line parameters and does not receive them. The code snippet
below demonstrates that if you don't receive any arguments you can
provide a description of what your program does and/or expects to
run properly. The else statement at the end of the main function performs
a similar action, except that too many arguments have been provided. 

\begin{lstlisting}[language=make,showstringspaces=false,tabsize=4,frame=TB]
if( argc == 1 ) {
	printf("Describe what your program does here\n");
	printf("What arguments does your program require\n");
	return 0;
}
\end{lstlisting}

The main functionality of this program expects exactly 2 command-line
arguments, the first argument is a string and the second is an integer.
Certainly, more rigorous testing to ensure the proper types were provided
as an argument could be done. The two arguments are stored in argv{[}1{]}
and argv{[}2{]}, the program name is displayed from argv{[}0{]} to
reaffirm the program name. Any arguments passed to your program are
stored as strings, thus for our second argument, it must be converted
to an integer via the to function. Once your program has received
the correct number of arguments and validated the arguments you can
use them throughout your program as any other variable.

\section*{Additional Information \label{sec:Additional-Information}}
\begin{itemize}
\item \url{http://www.cprogramming.com/tutorial/c/lesson14.html}
\item \url{http://fresh2refresh.com/c-programming/c-function/c-command-line-arguments/}
\item \url{http://stackoverflow.com/questions/16869467/command-line-arguments-reading-a-file}
\end{itemize}

\end{document}
